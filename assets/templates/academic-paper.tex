\documentclass[12pt,a4paper]{article}
\usepackage[utf8]{inputenc}
\usepackage[T1]{fontenc}
\usepackage{geometry}
\usepackage{amsmath,amssymb}
\usepackage{graphicx}
\usepackage{hyperref}
\usepackage{xcolor}
\usepackage{booktabs}
\usepackage{tabularx}
\usepackage{float}
\usepackage{caption}
\usepackage{subcaption}
\usepackage{enumitem}
\usepackage{setspace}
\usepackage{natbib}

% Page setup
\geometry{a4paper, margin=1in}
\onehalfspacing

% Hyperlinks
\hypersetup{
    colorlinks=true,
    linkcolor=blue!60!black,
    citecolor=green!50!black,
    urlcolor=blue!60!black
}

\title{\textbf{Paper Title: A Study of Something Important}}
\author{
    Author One\thanks{Department of Computer Science, University A. Email: author1@university.edu} \and
    Author Two\thanks{Department of Mathematics, University B. Email: author2@university.edu}
}
\date{\today}

\begin{document}

\maketitle

% --- ABSTRACT ---
\begin{abstract}
\noindent
This paper presents [brief description of the work]. We propose [method/approach] for [problem]. Our experiments on [dataset/benchmark] demonstrate [key result]. The results show [improvement/finding], achieving [specific metric] compared to [baseline]. Our code is available at [URL].

\vspace{0.5em}
\noindent
\textbf{Keywords:} keyword one, keyword two, keyword three, keyword four
\end{abstract}

% --- INTRODUCTION ---
\section{Introduction}
\label{sec:intro}

The problem of [topic] has received significant attention in recent years \citep{author2024}. Despite progress in [area], challenges remain in [specific challenge].

In this paper, we make the following contributions:
\begin{itemize}[itemsep=0.3em]
    \item We propose [contribution 1].
    \item We demonstrate [contribution 2].
    \item We achieve [contribution 3] on [benchmark].
\end{itemize}

% --- RELATED WORK ---
\section{Related Work}
\label{sec:related}

\paragraph{Topic Area 1.}
Prior work in [area] includes [brief survey]. \citet{author2023} proposed [method], while \citet{author2022} introduced [approach].

\paragraph{Topic Area 2.}
The field of [area] has explored [approaches]. Our work differs in [key distinction].

% --- METHOD ---
\section{Method}
\label{sec:method}

\subsection{Problem Formulation}

Given input $x \in \mathbb{R}^n$, our goal is to find $f^* = \arg\min_{f \in \mathcal{F}} \mathcal{L}(f(x), y)$, where $\mathcal{L}$ is the loss function and $y$ is the target.

\subsection{Proposed Approach}

Our method consists of two stages:

\begin{enumerate}[itemsep=0.3em]
    \item \textbf{Stage 1}: Description of first stage.
    \item \textbf{Stage 2}: Description of second stage.
\end{enumerate}

The key equation governing our approach is:
\begin{equation}
    \label{eq:main}
    \mathcal{L}(f) = \frac{1}{N} \sum_{i=1}^{N} \ell(f(x_i), y_i) + \lambda \|f\|^2
\end{equation}

where $\lambda > 0$ is the regularization parameter.

% --- EXPERIMENTS ---
\section{Experiments}
\label{sec:experiments}

\subsection{Experimental Setup}

We evaluate our method on [dataset] with [number] samples. All experiments use [hardware/software]. We compare against [baselines].

\subsection{Results}

Table~\ref{tab:results} summarizes our main results.

\begin{table}[htbp]
\centering
\caption{Comparison of methods on [benchmark]. Best results in \textbf{bold}.}
\label{tab:results}
\begin{tabular}{lccc}
\toprule
\textbf{Method} & \textbf{Accuracy} & \textbf{F1 Score} & \textbf{Runtime (s)} \\
\midrule
Baseline A & 85.2 & 83.1 & 12.3 \\
Baseline B & 87.5 & 85.4 & 15.7 \\
Ours & \textbf{91.3} & \textbf{89.8} & 14.2 \\
\bottomrule
\end{tabular}
\end{table}

As shown in Table~\ref{tab:results}, our method achieves [improvement] over the best baseline.

% Placeholder for figure:
% \begin{figure}[htbp]
%   \centering
%   \includegraphics[width=0.8\textwidth]{results_plot.png}
%   \caption{Performance comparison across different settings.}
%   \label{fig:results}
% \end{figure}

\subsection{Ablation Study}

To understand the contribution of each component, we conduct an ablation study (Table~\ref{tab:ablation}).

\begin{table}[htbp]
\centering
\caption{Ablation study results.}
\label{tab:ablation}
\begin{tabular}{lcc}
\toprule
\textbf{Configuration} & \textbf{Accuracy} & \textbf{$\Delta$} \\
\midrule
Full model & \textbf{91.3} & -- \\
Without component A & 89.1 & $-2.2$ \\
Without component B & 88.7 & $-2.6$ \\
Without both & 86.4 & $-4.9$ \\
\bottomrule
\end{tabular}
\end{table}

% --- CONCLUSION ---
\section{Conclusion}
\label{sec:conclusion}

We presented [method] for [problem]. Our approach achieves [result] on [benchmark], representing a [improvement] over prior work. Future work includes [direction 1] and [direction 2].

% --- REFERENCES ---
% To use: create a references.bib file and compile with bibtex
% \bibliographystyle{plainnat}
% \bibliography{references}

% Placeholder references (remove when using .bib file)
\begin{thebibliography}{9}
\bibitem[Author et~al.(2024)]{author2024}
Author, A., Author, B., and Author, C. (2024).
\newblock Title of the referenced paper.
\newblock \emph{Journal Name}, 1(1):1--10.

\bibitem[Author and Author(2023)]{author2023}
Author, D. and Author, E. (2023).
\newblock Another referenced paper title.
\newblock In \emph{Conference Name}, pages 100--110.

\bibitem[Author(2022)]{author2022}
Author, F. (2022).
\newblock A third referenced paper.
\newblock \emph{Journal Name}, 5(2):50--65.
\end{thebibliography}

\end{document}
