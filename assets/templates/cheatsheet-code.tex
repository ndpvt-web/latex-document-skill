% ============================================================================
% PROGRAMMING / CODE REFERENCE CARD TEMPLATE
% ============================================================================
% Landscape, 4-column layout optimized for command/syntax reference
% Paper: Letter/A4 landscape | Columns: 4 | Base font: 7pt
% Focus: Code snippets, command syntax, keyboard shortcuts, API reference
% Syntax highlighting with listings package
% Compile: pdflatex (auto-detected by compile script)
% ============================================================================

\documentclass[7pt,landscape]{extarticle}

% --- GEOMETRY ---
\usepackage[
    landscape,
    margin=4mm,
    top=5mm,
    bottom=3mm
]{geometry}

% --- ENCODING & FONTS ---
\usepackage[utf8]{inputenc}
\usepackage[T1]{fontenc}
\usepackage{lmodern}
\usepackage{microtype}

% --- LAYOUT ---
\usepackage{multicol}
\usepackage{enumitem}

% --- VISUAL ---
\usepackage[table]{xcolor}
\usepackage{tcolorbox}
\tcbuselibrary{breakable,skins,listings}
\usepackage{tabularx}
\usepackage{booktabs}
\usepackage{array}

% --- CODE ---
\usepackage{listings}

% --- MATH (light, for Big-O etc.) ---
\usepackage{amsmath}

% ============================================================================
% COLOR SCHEME — Dark header, syntax-highlight-friendly
% ============================================================================
\definecolor{cdPrimary}{RGB}{0,122,204}         % VS Code blue
\definecolor{cdSecondary}{RGB}{30,30,30}         % Dark header bar
\definecolor{cdAccent1}{RGB}{215,58,73}          % Red (warnings/destructive)
\definecolor{cdAccent2}{RGB}{40,167,69}          % Green (success/safe)
\definecolor{cdAccent3}{RGB}{227,98,9}           % Orange (deprecated/caution)
\definecolor{cdLightBg}{RGB}{246,248,250}        % GitHub light bg
\definecolor{cdCodeBg}{RGB}{250,251,252}         % Code block bg
\definecolor{cdBorder}{RGB}{208,215,222}         % Subtle borders

% --- Syntax colors ---
\definecolor{synKeyword}{RGB}{0,0,200}
\definecolor{synString}{RGB}{163,21,21}
\definecolor{synComment}{RGB}{106,115,125}
\definecolor{synNumber}{RGB}{0,128,128}

% ============================================================================
% GLOBAL SPACING
% ============================================================================
\pagestyle{empty}
\setlength{\parindent}{0pt}
\setlength{\parskip}{0pt}
\setlength{\columnsep}{2.5mm}
\setlength{\columnseprule}{0.15pt}
\def\columnseprulecolor{\color{cdBorder}}
\setlength{\premulticols}{0pt}
\setlength{\postmulticols}{0pt}
\setlength{\multicolsep}{0pt}

% --- Ultra-compact lists ---
\setlist{nosep, leftmargin=*, topsep=0pt, partopsep=0pt, parsep=0pt, itemsep=0.2pt}
\setlist[itemize]{leftmargin=2.5mm, label={\textcolor{cdPrimary}{\scriptsize\textbullet}}}

% --- Tight line spacing ---
\linespread{0.92}

% ============================================================================
% CUSTOM ENVIRONMENTS
% ============================================================================

% --- Section header (dark banner) ---
\newtcolorbox{codesection}[1]{%
    colback=cdPrimary,
    colframe=cdPrimary,
    coltext=white,
    fontupper=\bfseries\fontsize{7pt}{8pt}\selectfont\centering,
    boxrule=0pt,
    arc=0.8mm,
    top=0.3mm, bottom=0.3mm,
    left=1mm, right=1mm,
    before skip=0.8mm, after skip=0.4mm,
    halign=center,
    after upper={\par\noindent #1\par\vspace{-2pt}}
}

% --- Command reference box (titled, for tables/lists) ---
\newtcolorbox{cmdbox}[2][]{%
    colback=cdLightBg,
    colframe=cdBorder,
    fonttitle=\bfseries\fontsize{6.5pt}{7.5pt}\selectfont,
    title=#2,
    boxrule=0.3pt,
    arc=0.5mm,
    top=0.3mm, bottom=0.3mm,
    left=0.8mm, right=0.8mm,
    toptitle=0.2mm, bottomtitle=0.2mm,
    before skip=0.4mm, after skip=0.4mm,
    #1
}

% --- Code block box (for syntax examples) ---
\newtcolorbox{codeblock}[2][]{%
    colback=cdCodeBg,
    colframe=cdBorder,
    fonttitle=\bfseries\fontsize{6pt}{7pt}\selectfont\ttfamily,
    title=#2,
    boxrule=0.25pt,
    arc=0.3mm,
    top=0.2mm, bottom=0.2mm,
    left=0.5mm, right=0.5mm,
    toptitle=0.2mm, bottomtitle=0.2mm,
    before skip=0.4mm, after skip=0.4mm,
    #1
}

% --- Danger/warning box ---
\newtcolorbox{dangerbox}[1][]{%
    colback=cdAccent1!5,
    colframe=cdAccent1!80!black,
    boxrule=0.3pt,
    arc=0.3mm,
    top=0.3mm, bottom=0.3mm,
    left=0.8mm, right=0.8mm,
    before skip=0.4mm, after skip=0.4mm,
    #1
}

% --- Tip/success box ---
\newtcolorbox{safebox}[1][]{%
    colback=cdAccent2!5,
    colframe=cdAccent2!80!black,
    boxrule=0.3pt,
    arc=0.3mm,
    top=0.3mm, bottom=0.3mm,
    left=0.8mm, right=0.8mm,
    before skip=0.4mm, after skip=0.4mm,
    #1
}

% ============================================================================
% CUSTOM COMMANDS
% ============================================================================
\newcommand{\code}[1]{\texttt{\fontsize{5.5pt}{6.5pt}\selectfont#1}}
\newcommand{\cmd}[1]{\textcolor{cdPrimary}{\texttt{\textbf{#1}}}}
\newcommand{\key}[1]{\colorbox{cdLightBg}{\texttt{\fontsize{5.5pt}{6.5pt}\selectfont#1}}}
\newcommand{\flag}[1]{\textcolor{cdAccent3}{\texttt{#1}}}

% ============================================================================
% CODE LISTING STYLE
% ============================================================================
\lstdefinestyle{refcode}{%
    basicstyle=\ttfamily\fontsize{5.5pt}{6.5pt}\selectfont,
    backgroundcolor=\color{cdCodeBg},
    frame=none,
    numbers=none,
    breaklines=true,
    breakatwhitespace=true,
    tabsize=2,
    columns=flexible,
    keepspaces=true,
    xleftmargin=0.5mm,
    xrightmargin=0.5mm,
    aboveskip=0.5mm,
    belowskip=0.5mm,
    keywordstyle=\color{synKeyword}\bfseries,
    commentstyle=\color{synComment}\itshape,
    stringstyle=\color{synString},
    numberstyle=\color{synNumber}
}
\lstset{style=refcode}

% ============================================================================
% DOCUMENT
% ============================================================================
\begin{document}

% --- TITLE BAR ---
\noindent\colorbox{cdSecondary}{%
    \parbox{\dimexpr\textwidth-2\fboxsep\relax}{%
        \centering\color{white}%
        \fontsize{9pt}{11pt}\selectfont\textbf{PYTHON QUICK REFERENCE CARD}%
        \hfill%
        \fontsize{6pt}{7pt}\selectfont Python 3.12+ \textbar{} v1.0 \textbar{} 2026%
    }%
}

\vspace{0.5mm}

% ============================================================================
% PAGE 1
% ============================================================================
\begin{multicols}{4}

% --- BASICS ---
\begin{codesection}{Data Types}\end{codesection}

\begin{cmdbox}{Primitive Types}
{\fontsize{5.5pt}{6.5pt}\selectfont
\renewcommand{\arraystretch}{0.95}
\begin{tabular}{@{}ll@{}}
\cmd{int} & \code{42}, \code{0xff}, \code{0b1010} \\
\cmd{float} & \code{3.14}, \code{1e-3}, \code{float('inf')} \\
\cmd{str} & \code{'hello'}, \code{f"\{x\}"} \\
\cmd{bool} & \code{True}, \code{False} \\
\cmd{None} & null/nil equivalent \\
\cmd{complex} & \code{3+4j} \\
\end{tabular}
}
\end{cmdbox}

\begin{cmdbox}{Collections}
{\fontsize{5.5pt}{6.5pt}\selectfont
\renewcommand{\arraystretch}{0.95}
\begin{tabular}{@{}ll@{}}
\cmd{list} & \code{[1,2,3]} mutable, ordered \\
\cmd{tuple} & \code{(1,2,3)} immutable \\
\cmd{dict} & \code{\{'a':1\}} key-value \\
\cmd{set} & \code{\{1,2,3\}} unique vals \\
\cmd{frozenset} & immutable set \\
\cmd{deque} & double-ended queue \\
\end{tabular}
}
\end{cmdbox}

% --- STRINGS ---
\begin{codesection}{String Operations}\end{codesection}

\begin{cmdbox}{Common Methods}
{\fontsize{5.5pt}{6.5pt}\selectfont
\renewcommand{\arraystretch}{0.95}
\begin{tabular}{@{}lp{2.8cm}@{}}
\cmd{.split(sep)} & Split by delimiter \\
\cmd{.join(iter)} & Join iterable \\
\cmd{.strip()} & Remove whitespace \\
\cmd{.replace(a,b)} & Replace substring \\
\cmd{.find(sub)} & Find index (-1 miss) \\
\cmd{.startswith(s)} & Prefix check \\
\cmd{.upper()/.lower()} & Case conversion \\
\cmd{.format()} & String formatting \\
\end{tabular}
}
\end{cmdbox}

\begin{codeblock}{f-string Formatting}
\begin{lstlisting}[language=Python]
f"{x:.2f}"     # 2 decimals
f"{x:>10}"     # right-align 10
f"{x:#06x}"    # hex 0x00ff
f"{d:%Y-%m-%d}" # date format
f"{n:,}"       # thousands sep
\end{lstlisting}
\end{codeblock}

% --- LISTS ---
\begin{codesection}{List Operations}\end{codesection}

\begin{cmdbox}{Methods}
{\fontsize{5.5pt}{6.5pt}\selectfont
\renewcommand{\arraystretch}{0.95}
\begin{tabular}{@{}ll@{}}
\cmd{.append(x)} & Add to end \\
\cmd{.extend(iter)} & Append iterable \\
\cmd{.insert(i, x)} & Insert at index \\
\cmd{.pop(i)} & Remove by index \\
\cmd{.remove(x)} & Remove first match \\
\cmd{.sort(key=)} & In-place sort \\
\cmd{.reverse()} & In-place reverse \\
\cmd{.index(x)} & Find index \\
\cmd{.count(x)} & Count occurrences \\
\end{tabular}
}
\end{cmdbox}

\begin{codeblock}{Slicing \& Comprehensions}
\begin{lstlisting}[language=Python]
a[1:5]     # index 1 to 4
a[::2]     # every 2nd element
a[::-1]    # reversed copy
[x**2 for x in range(10)]
[x for x in a if x > 0]
{k:v for k,v in d.items()}
\end{lstlisting}
\end{codeblock}

% --- DICTS ---
\begin{codesection}{Dict Operations}\end{codesection}

\begin{cmdbox}{Methods}
{\fontsize{5.5pt}{6.5pt}\selectfont
\renewcommand{\arraystretch}{0.95}
\begin{tabular}{@{}ll@{}}
\cmd{.get(k, def)} & Safe lookup \\
\cmd{.keys()} & All keys \\
\cmd{.values()} & All values \\
\cmd{.items()} & Key-value pairs \\
\cmd{.update(d2)} & Merge dicts \\
\cmd{.pop(k)} & Remove + return \\
\cmd{.setdefault(k,v)} & Get or set \\
\cmd{d1 | d2} & Merge (3.9+) \\
\end{tabular}
}
\end{cmdbox}

% --- CONTROL FLOW ---
\begin{codesection}{Control Flow}\end{codesection}

\begin{codeblock}{Conditionals}
\begin{lstlisting}[language=Python]
if x > 0:
    pass
elif x == 0:
    pass
else:
    pass

# Ternary
y = a if cond else b

# Match (3.10+)
match cmd:
    case "q": quit()
    case "h": help()
    case _: unknown()
\end{lstlisting}
\end{codeblock}

\begin{codeblock}{Loops}
\begin{lstlisting}[language=Python]
for x in iterable:
    if skip: continue
    if done: break
else:  # no break

while cond:
    pass

# Enumerate
for i, v in enumerate(lst):

# Zip
for a, b in zip(l1, l2):
\end{lstlisting}
\end{codeblock}

% --- FUNCTIONS ---
\begin{codesection}{Functions}\end{codesection}

\begin{codeblock}{Definition \& Args}
\begin{lstlisting}[language=Python]
def f(x, y=10, *args, **kw):
    return x + y

# Lambda
sq = lambda x: x**2

# Type hints
def add(a: int, b: int) -> int:
    return a + b

# Decorators
@decorator
def func(): pass
\end{lstlisting}
\end{codeblock}

\begin{cmdbox}{Built-in Functions}
{\fontsize{5.5pt}{6.5pt}\selectfont
\renewcommand{\arraystretch}{0.95}
\begin{tabular}{@{}ll@{}}
\cmd{map(f, iter)} & Apply f to each \\
\cmd{filter(f, iter)} & Keep if f(x) true \\
\cmd{sorted(iter)} & New sorted list \\
\cmd{zip(*iters)} & Parallel iterate \\
\cmd{any()/all()} & Boolean reduce \\
\cmd{isinstance(x,T)} & Type check \\
\end{tabular}
}
\end{cmdbox}

% --- FILE I/O ---
\begin{codesection}{File I/O}\end{codesection}

\begin{codeblock}{Reading \& Writing}
\begin{lstlisting}[language=Python]
# Read
with open('f.txt') as f:
    text = f.read()
    lines = f.readlines()

# Write
with open('f.txt','w') as f:
    f.write("hello\n")

# JSON
import json
d = json.load(open('f.json'))
json.dump(d, open('o.json','w'))
\end{lstlisting}
\end{codeblock}

% --- ERROR HANDLING ---
\begin{codesection}{Error Handling}\end{codesection}

\begin{codeblock}{try / except / finally}
\begin{lstlisting}[language=Python]
try:
    risky()
except ValueError as e:
    handle(e)
except (KeyError, IndexError):
    fallback()
else:
    success()  # no exception
finally:
    cleanup()  # always runs

# Raise
raise ValueError("msg")
\end{lstlisting}
\end{codeblock}

% --- CLASSES ---
\begin{codesection}{Classes \& OOP}\end{codesection}

\begin{codeblock}{Class Definition}
\begin{lstlisting}[language=Python]
class Dog(Animal):
    def __init__(self, name):
        super().__init__()
        self.name = name

    def speak(self) -> str:
        return f"{self.name} barks"

    @property
    def info(self):
        return self._info

    @classmethod
    def create(cls, name):
        return cls(name)

    @staticmethod
    def species():
        return "Canis"
\end{lstlisting}
\end{codeblock}

\begin{cmdbox}{Dunder Methods}
{\fontsize{5.5pt}{6.5pt}\selectfont
\renewcommand{\arraystretch}{0.95}
\begin{tabular}{@{}ll@{}}
\cmd{\_\_init\_\_} & Constructor \\
\cmd{\_\_str\_\_} & String repr \\
\cmd{\_\_repr\_\_} & Debug repr \\
\cmd{\_\_len\_\_} & \code{len(obj)} \\
\cmd{\_\_getitem\_\_} & \code{obj[key]} \\
\cmd{\_\_iter\_\_} & \code{for x in obj} \\
\cmd{\_\_eq\_\_} & \code{obj == other} \\
\cmd{\_\_hash\_\_} & Hash value \\
\cmd{\_\_enter/exit\_\_} & Context mgr \\
\end{tabular}
}
\end{cmdbox}

\end{multicols}

% ============================================================================
% PAGE 2
% ============================================================================
\newpage

\noindent\colorbox{cdSecondary}{%
    \parbox{\dimexpr\textwidth-2\fboxsep\relax}{%
        \centering\color{white}%
        \fontsize{9pt}{11pt}\selectfont\textbf{PYTHON REFERENCE -- PAGE 2}%
        \hfill%
        \fontsize{6pt}{7pt}\selectfont Standard Library \textbar{} Common Patterns%
    }%
}

\vspace{0.5mm}

\begin{multicols}{4}

% --- STDLIB ---
\begin{codesection}{Standard Library}\end{codesection}

\begin{cmdbox}{os / pathlib}
{\fontsize{5.5pt}{6.5pt}\selectfont
\renewcommand{\arraystretch}{0.95}
\begin{tabular}{@{}lp{2.5cm}@{}}
\cmd{Path('f').read\_text()} & Read file \\
\cmd{Path('d').mkdir()} & Create dir \\
\cmd{Path('f').exists()} & Check exists \\
\cmd{Path('f').stem} & Filename no ext \\
\cmd{os.environ['KEY']} & Env variable \\
\cmd{os.getcwd()} & Working dir \\
\end{tabular}
}
\end{cmdbox}

\begin{cmdbox}{itertools}
{\fontsize{5.5pt}{6.5pt}\selectfont
\renewcommand{\arraystretch}{0.95}
\begin{tabular}{@{}ll@{}}
\cmd{chain(*iters)} & Flatten iterables \\
\cmd{product(a,b)} & Cartesian product \\
\cmd{combinations(a,n)} & n-choose combos \\
\cmd{permutations(a,n)} & Ordered combos \\
\cmd{groupby(iter,key)} & Group by key \\
\cmd{islice(iter,n)} & Slice iterator \\
\end{tabular}
}
\end{cmdbox}

\begin{cmdbox}{collections}
{\fontsize{5.5pt}{6.5pt}\selectfont
\renewcommand{\arraystretch}{0.95}
\begin{tabular}{@{}ll@{}}
\cmd{Counter(iter)} & Count elements \\
\cmd{defaultdict(type)} & Default values \\
\cmd{OrderedDict()} & Insertion order \\
\cmd{namedtuple(n,flds)} & Named fields \\
\cmd{deque()} & Fast append/pop \\
\end{tabular}
}
\end{cmdbox}

\begin{cmdbox}{re (Regular Expressions)}
{\fontsize{5.5pt}{6.5pt}\selectfont
\renewcommand{\arraystretch}{0.95}
\begin{tabular}{@{}ll@{}}
\cmd{re.search(p, s)} & First match \\
\cmd{re.findall(p, s)} & All matches \\
\cmd{re.sub(p, r, s)} & Replace \\
\cmd{re.split(p, s)} & Split by pattern \\
\cmd{re.compile(p)} & Precompile \\
\end{tabular}
}
\end{cmdbox}

% --- ASYNC ---
\begin{codesection}{Async / Concurrency}\end{codesection}

\begin{codeblock}{asyncio}
\begin{lstlisting}[language=Python]
import asyncio

async def fetch(url):
    await asyncio.sleep(1)
    return data

async def main():
    results = await asyncio.gather(
        fetch(u1), fetch(u2)
    )

asyncio.run(main())
\end{lstlisting}
\end{codeblock}

\begin{codeblock}{Threading / Multiprocessing}
\begin{lstlisting}[language=Python]
from concurrent.futures import (
    ThreadPoolExecutor,
    ProcessPoolExecutor
)

with ThreadPoolExecutor(4) as ex:
    results = ex.map(fn, data)

with ProcessPoolExecutor() as ex:
    futures = [ex.submit(fn, x)
               for x in data]
\end{lstlisting}
\end{codeblock}

% --- TESTING ---
\begin{codesection}{Testing}\end{codesection}

\begin{codeblock}{pytest}
\begin{lstlisting}[language=Python]
def test_add():
    assert add(1,2) == 3

@pytest.fixture
def db():
    return setup_db()

@pytest.mark.parametrize(
    "x,y,z", [(1,2,3),(4,5,9)]
)
def test_sum(x, y, z):
    assert x + y == z
\end{lstlisting}
\end{codeblock}

% --- COMMON PATTERNS ---
\begin{codesection}{Common Patterns}\end{codesection}

\begin{codeblock}{Context Managers}
\begin{lstlisting}[language=Python]
from contextlib import (
    contextmanager
)

@contextmanager
def timer():
    t0 = time.time()
    yield
    print(f"{time.time()-t0:.2f}s")

with timer():
    expensive_op()
\end{lstlisting}
\end{codeblock}

\begin{codeblock}{Dataclasses}
\begin{lstlisting}[language=Python]
from dataclasses import (
    dataclass, field
)

@dataclass
class Point:
    x: float
    y: float
    label: str = ""
    tags: list = field(
        default_factory=list
    )

    def dist(self, other):
        return ((self.x-other.x)**2
              + (self.y-other.y)**2)**0.5
\end{lstlisting}
\end{codeblock}

\begin{codeblock}{Type Hints}
\begin{lstlisting}[language=Python]
from typing import (
    Optional, Union, TypeAlias
)

Vector: TypeAlias = list[float]

def process(
    data: list[dict[str, int]],
    flag: bool = False,
) -> Optional[str]:
    ...

# Python 3.10+ union
def f(x: int | str) -> None: ...
\end{lstlisting}
\end{codeblock}

% --- KEYBOARD SHORTCUTS ---
\begin{codesection}{Useful Shortcuts}\end{codesection}

\begin{cmdbox}{REPL / IPython}
{\fontsize{5.5pt}{6.5pt}\selectfont
\renewcommand{\arraystretch}{0.95}
\begin{tabular}{@{}ll@{}}
\key{Tab} & Autocomplete \\
\key{Ctrl+D} & Exit \\
\code{\_} & Last result \\
\code{help(obj)} & Documentation \\
\code{dir(obj)} & List attributes \\
\code{type(obj)} & Object type \\
\code{vars(obj)} & Instance dict \\
\end{tabular}
}
\end{cmdbox}

\begin{safebox}
\textbf{Pro tip:} Use \code{python -m pdb script.py} for debugging. \code{breakpoint()} in code drops into debugger.
\end{safebox}

% --- Footer ---
\vfill
\begin{center}
{\fontsize{4.5pt}{5.5pt}\selectfont\textcolor{cdBorder}{%
Programming Reference Card Template \textbar{} Customize for any language%
}}
\end{center}

\end{multicols}

\end{document}
