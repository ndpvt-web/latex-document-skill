% ============================================================================
% EXAM FORMULA SHEET / CHEAT SHEET TEMPLATE
% ============================================================================
% Portrait, 2-column layout optimized for MAXIMUM density on exams
% Paper: Letter/A4 portrait | Columns: 2 | Base font: 6pt
% Focus: Mathematical formulas, theorems, definitions, procedures
% B&W printing friendly — minimal color usage (grayscale only)
% Front-and-back (2 pages) with student/course header
% Compile: pdflatex (auto-detected by compile script)
% ============================================================================

\documentclass[6pt]{extarticle}  % extarticle supports 6pt base font

% --- GEOMETRY: Ultra-tight margins ---
\usepackage[
    margin=5mm,
    top=8mm,            % Slightly more top for header
    bottom=4mm
]{geometry}

% --- ENCODING & FONTS ---
\usepackage[utf8]{inputenc}
\usepackage[T1]{fontenc}
\usepackage{lmodern}
\usepackage{microtype}          % Protrusion + expansion for density

% --- LAYOUT ---
\usepackage{multicol}
\usepackage{enumitem}

% --- VISUAL ---
\usepackage{xcolor}
\usepackage{tcolorbox}
\tcbuselibrary{breakable,skins}

% --- MATH (comprehensive) ---
\usepackage{amsmath,amssymb,mathtools,amsthm}
\usepackage{array}
\usepackage{booktabs}

% ============================================================================
% GRAYSCALE COLOR SCHEME — Prints perfectly on any B&W printer
% ============================================================================
\definecolor{exDark}{RGB}{50,50,50}         % Section headers
\definecolor{exMed}{RGB}{110,110,110}       % Borders, accents
\definecolor{exLight}{RGB}{235,235,235}     % Box backgrounds
\definecolor{exRule}{RGB}{180,180,180}      % Column separator

% ============================================================================
% GLOBAL SPACING — Maximum density for exam use
% ============================================================================
\pagestyle{empty}
\setlength{\parindent}{0pt}
\setlength{\parskip}{0pt}
\setlength{\columnsep}{3mm}
\setlength{\columnseprule}{0.1pt}
\def\columnseprulecolor{\color{exRule}}
\setlength{\premulticols}{0pt}
\setlength{\postmulticols}{0pt}
\setlength{\multicolsep}{0pt}

% --- Ultra-compact lists ---
\setlist{nosep, leftmargin=*, topsep=0pt, partopsep=0pt, parsep=0pt, itemsep=0.2pt}
\setlist[itemize]{leftmargin=2.5mm, label={\tiny\textbullet}}
\setlist[enumerate]{leftmargin=3mm}

% --- Minimal math spacing ---
\AtBeginDocument{%
    \setlength{\abovedisplayskip}{1pt}
    \setlength{\belowdisplayskip}{1pt}
    \setlength{\abovedisplayshortskip}{0pt}
    \setlength{\belowdisplayshortskip}{0pt}
    \setlength{\jot}{1pt}
}

% --- Tight line spacing ---
\linespread{0.9}

% ============================================================================
% CUSTOM ENVIRONMENTS — Exam-focused, minimal chrome
% ============================================================================

% --- Section header (dark bar, white text, sharp corners) ---
\newcommand{\sheetsection}[1]{%
    \vspace{0.5mm}%
    \noindent\colorbox{exDark}{%
        \parbox{\dimexpr\linewidth-2\fboxsep\relax}{%
            \centering\color{white}\bfseries\fontsize{7pt}{8pt}\selectfont #1%
        }%
    }%
    \vspace{0.3mm}%
}

% --- Subsection header (bold, gray underline) ---
\newcommand{\sheetsubsection}[1]{%
    \vspace{0.3mm}%
    \noindent{\color{exDark}\bfseries\fontsize{6.5pt}{7.5pt}\selectfont #1}%
    \par\vspace{-0.5mm}\noindent\textcolor{exRule}{\rule{\linewidth}{0.2pt}}%
    \vspace{0.2mm}%
}

% --- Theorem box (most important results) ---
\newtcolorbox{thmbox}[1]{%
    colback=white,
    colframe=exDark,
    fonttitle=\bfseries\fontsize{6pt}{7pt}\selectfont,
    title=\textsc{Thm:} #1,
    boxrule=0.4pt,
    arc=0mm,            % Sharp corners — saves space
    top=0.3mm, bottom=0.3mm,
    left=0.8mm, right=0.8mm,
    toptitle=0.2mm, bottomtitle=0.2mm,
    before skip=0.5mm, after skip=0.5mm
}

% --- Definition box (key concepts) ---
\newtcolorbox{defbox}[1]{%
    colback=exLight,
    colframe=exMed,
    fonttitle=\bfseries\fontsize{6pt}{7pt}\selectfont,
    title=\textsc{Def:} #1,
    boxrule=0.3pt,
    arc=0mm,
    top=0.3mm, bottom=0.3mm,
    left=0.8mm, right=0.8mm,
    toptitle=0.2mm, bottomtitle=0.2mm,
    before skip=0.5mm, after skip=0.5mm
}

% --- Formula box (important equations, no title bar) ---
\newtcolorbox{formulabox}[1]{%
    colback=white,
    colframe=exMed,
    fonttitle=\bfseries\fontsize{6pt}{7pt}\selectfont,
    title=#1,
    boxrule=0.25pt,
    arc=0mm,
    top=0.2mm, bottom=0.2mm,
    left=0.8mm, right=0.8mm,
    toptitle=0.2mm, bottomtitle=0.2mm,
    before skip=0.4mm, after skip=0.4mm
}

% --- Procedure / algorithm box ---
\newtcolorbox{procbox}[1]{%
    colback=white,
    colframe=exDark,
    fonttitle=\bfseries\fontsize{6pt}{7pt}\selectfont,
    title=\textsc{Procedure:} #1,
    boxrule=0.3pt,
    arc=0mm,
    top=0.3mm, bottom=0.3mm,
    left=0.8mm, right=0.8mm,
    toptitle=0.2mm, bottomtitle=0.2mm,
    before skip=0.4mm, after skip=0.4mm
}

% ============================================================================
% CUSTOM COMMANDS — Math shortcuts for density
% ============================================================================
\DeclareMathOperator{\rank}{rank}
\DeclareMathOperator{\nullity}{nullity}
\DeclareMathOperator{\Span}{span}
\DeclareMathOperator{\im}{im}
\DeclareMathOperator{\tr}{tr}
\DeclareMathOperator{\diag}{diag}
\DeclareMathOperator{\sgn}{sgn}
\newcommand{\vect}[1]{\mathbf{#1}}
\newcommand{\mat}[1]{\mathbf{#1}}
\newcommand{\R}{\mathbb{R}}
\newcommand{\C}{\mathbb{C}}
\newcommand{\N}{\mathbb{N}}
\newcommand{\Z}{\mathbb{Z}}
\newcommand{\Q}{\mathbb{Q}}
\newcommand{\norm}[1]{\left\|#1\right\|}
\newcommand{\abs}[1]{\left|#1\right|}
\newcommand{\pd}[2]{\frac{\partial #1}{\partial #2}}

% ============================================================================
% DOCUMENT BEGINS
% ============================================================================
\begin{document}

% --- HEADER: Course info + student name ---
\begin{center}
\fontsize{8pt}{9pt}\selectfont
\textbf{MATH 401: Advanced Calculus \& Linear Algebra}\\
{\fontsize{6pt}{7pt}\selectfont Exam Formula Sheet \textbar{} Midterm -- Spring 2026 \textbar{} Student: \underline{\hspace{3cm}}}
\end{center}

\vspace{0.3mm}
\noindent\rule{\textwidth}{0.3pt}
\vspace{0.3mm}

% ============================================================================
% PAGE 1 — FRONT SIDE
% ============================================================================
\begin{multicols}{2}

\sheetsection{Differential Calculus}
\begin{defbox}{Derivative}
$f'(x) = \lim_{h \to 0} \dfrac{f(x+h) - f(x)}{h}$
\end{defbox}
\begin{formulabox}{Common Derivatives}
\vspace{-1mm}
\begin{align*}
\textstyle\frac{d}{dx}(x^n) &= nx^{n-1} & \textstyle\frac{d}{dx}(e^x) &= e^x \\
\textstyle\frac{d}{dx}(\ln x) &= \frac{1}{x} & \textstyle\frac{d}{dx}(\sin x) &= \cos x \\
\textstyle\frac{d}{dx}(\cos x) &= -\sin x & \textstyle\frac{d}{dx}(\tan x) &= \sec^2 x \\
\textstyle\frac{d}{dx}(\arcsin x) &= \frac{1}{\sqrt{1-x^2}} & \textstyle\frac{d}{dx}(\arctan x) &= \frac{1}{1+x^2}
\end{align*}
\vspace{-2mm}
\end{formulabox}
\begin{thmbox}{Chain Rule}
$(f \circ g)'(x) = f'(g(x)) \cdot g'(x)$
\end{thmbox}
\begin{thmbox}{Product / Quotient Rules}
$(fg)' = f'g + fg'$ \hfill $\left(\frac{f}{g}\right)' = \frac{f'g - fg'}{g^2}$
\end{thmbox}
\begin{thmbox}{L'H\^opital's Rule}
If $\lim \frac{f}{g} = \frac{0}{0}$ or $\frac{\pm\infty}{\pm\infty}$, then
$\lim \frac{f(x)}{g(x)} = \lim \frac{f'(x)}{g'(x)}$
\end{thmbox}
\sheetsubsection{Taylor Series}
\begin{formulabox}{Taylor Expansion at $a$}
$f(x) = \sum_{n=0}^\infty \frac{f^{(n)}(a)}{n!}(x-a)^n$
\end{formulabox}
\begin{formulabox}{Common Series}
\vspace{-1mm}
{\fontsize{5.5pt}{6.5pt}\selectfont
\begin{align*}
e^x &= \sum \frac{x^n}{n!} = 1 + x + \frac{x^2}{2!} + \cdots \\
\sin x &= x - \frac{x^3}{3!} + \frac{x^5}{5!} - \cdots \\
\cos x &= 1 - \frac{x^2}{2!} + \frac{x^4}{4!} - \cdots \\
\ln(1+x) &= x - \frac{x^2}{2} + \frac{x^3}{3} - \cdots \\
\frac{1}{1-x} &= 1 + x + x^2 + x^3 + \cdots
\end{align*}
}
\vspace{-2mm}
\end{formulabox}
\sheetsection{Integral Calculus}
\begin{thmbox}{Fundamental Thm of Calculus}
If $F'(x) = f(x)$, then $\int_a^b f(x)\,dx = F(b) - F(a)$
\end{thmbox}
\begin{formulabox}{Common Integrals}
\vspace{-1mm}
{\fontsize{5.5pt}{6.5pt}\selectfont
\begin{align*}
\int x^n\,dx &= \frac{x^{n+1}}{n+1}+C\ (n\neq -1) &
\int \frac{1}{x}\,dx &= \ln|x|+C \\
\int e^x\,dx &= e^x+C &
\int \sin x\,dx &= -\cos x+C \\
\int \cos x\,dx &= \sin x+C &
\int \frac{dx}{1+x^2} &= \arctan x+C
\end{align*}
}
\vspace{-2mm}
\end{formulabox}
\begin{thmbox}{Integration by Parts}
$\int u\,dv = uv - \int v\,du$
\end{thmbox}
\begin{procbox}{Trig Substitution}
\begin{itemize}
\item $\sqrt{a^2-x^2}$: let $x = a\sin\theta$
\item $\sqrt{a^2+x^2}$: let $x = a\tan\theta$
\item $\sqrt{x^2-a^2}$: let $x = a\sec\theta$
\end{itemize}
\end{procbox}
\sheetsection{Multivariable Calculus}
\begin{defbox}{Gradient}
$\nabla f = \left(\pd{f}{x},\, \pd{f}{y},\, \pd{f}{z}\right)$
\end{defbox}
\begin{thmbox}{2nd Derivative Test}
$D = f_{xx}f_{yy} - (f_{xy})^2$ at critical pt.\\
$D>0,\, f_{xx}>0$: local min \hfill $D>0,\, f_{xx}<0$: local max\\
$D<0$: saddle \hfill $D=0$: inconclusive
\end{thmbox}
\begin{formulabox}{Lagrange Multipliers}
Optimize $f$ subject to $g = c$: Solve $\nabla f = \lambda\nabla g$ and $g = c$
\end{formulabox}
\sheetsubsection{Vector Calculus}
\begin{thmbox}{Green's Theorem}
$\oint_C (P\,dx + Q\,dy) = \iint_D \left(\pd{Q}{x} - \pd{P}{y}\right)dA$
\end{thmbox}
\begin{thmbox}{Divergence Theorem}
$\iiint_V \nabla\cdot\vect{F}\,dV = \iint_S \vect{F}\cdot\hat{\vect{n}}\,dS$
\end{thmbox}
\begin{thmbox}{Stokes' Theorem}
$\oint_C \vect{F}\cdot d\vect{r} = \iint_S (\nabla\times\vect{F})\cdot\hat{\vect{n}}\,dS$
\end{thmbox}
\sheetsection{Linear Algebra}
\begin{thmbox}{Dimension Theorem}
$\dim(V) = \rank(\mat{A}) + \nullity(\mat{A})$
\end{thmbox}
\begin{formulabox}{2$\times$2 Inverse}
$\mat{A}^{-1} = \frac{1}{ad-bc}\begin{pmatrix}d&-b\\-c&a\end{pmatrix}$
\end{formulabox}
\begin{formulabox}{Determinant Properties}
{\fontsize{5.5pt}{6.5pt}\selectfont
$\det(AB) = \det A\cdot\det B$ \hfill
$\det(A^T) = \det A$ \hfill
$\det(A^{-1}) = \frac{1}{\det A}$

Row swap $\Rightarrow$ sign change \hfill Row scale by $k$ $\Rightarrow$ det scales by $k$
}
\end{formulabox}
\sheetsubsection{Eigenvalues \& Eigenvectors}
\begin{procbox}{Finding Eigenvalues}
\begin{enumerate}
\item Solve char. eq.: $\det(\mat{A} - \lambda\mat{I}) = 0$
\item For each $\lambda$: solve $(\mat{A} - \lambda\mat{I})\vect{v} = \vect{0}$
\end{enumerate}
\end{procbox}
\begin{thmbox}{Spectral Theorem}
Symmetric $\mat{A}$: all real eigenvalues, orthogonal eigenvectors.
$\mat{A} = \mat{Q}\mat{\Lambda}\mat{Q}^T$ where $\mat{Q}$ orthogonal.
\end{thmbox}
\begin{formulabox}{Gram-Schmidt}
{\fontsize{5.5pt}{6.5pt}\selectfont
$\vect{u}_1 = \tfrac{\vect{v}_1}{\norm{\vect{v}_1}}$, \
$\vect{u}_k = \tfrac{\vect{v}_k - \sum_{i<k}\langle\vect{v}_k,\vect{u}_i\rangle\vect{u}_i}{\norm{\vect{v}_k - \sum_{i<k}\langle\vect{v}_k,\vect{u}_i\rangle\vect{u}_i}}$}
\end{formulabox}

\end{multicols}

% ============================================================================
% PAGE 2 — BACK SIDE
% ============================================================================
\newpage

\begin{center}
\fontsize{8pt}{9pt}\selectfont
\textbf{FORMULA SHEET -- PAGE 2}\\
{\fontsize{6pt}{7pt}\selectfont Differential Equations \textbar{} Probability \textbar{} Series \textbar{} Identities}
\end{center}

\vspace{0.3mm}
\noindent\rule{\textwidth}{0.3pt}
\vspace{0.3mm}

\begin{multicols}{2}

\sheetsection{Differential Equations}
\begin{formulabox}{Separable}
$\frac{dy}{dx} = g(x)h(y) \Rightarrow \int\frac{dy}{h(y)} = \int g(x)\,dx$
\end{formulabox}
\begin{formulabox}{First-Order Linear}
$y' + p(x)y = q(x)$. \quad IF: $\mu = e^{\int p\,dx}$.

$y = \frac{1}{\mu}\left(\int \mu\, q\,dx + C\right)$
\end{formulabox}
\begin{formulabox}{2nd-Order Constant Coeff}
$ay'' + by' + cy = 0$. \quad Char eq: $ar^2+br+c=0$.

\textbf{Two real:} $y = c_1 e^{r_1 x} + c_2 e^{r_2 x}$\\
\textbf{Repeated:} $y = (c_1+c_2 x)e^{rx}$\\
\textbf{Complex $\alpha\pm\beta i$:} $y = e^{\alpha x}(c_1\cos\beta x + c_2\sin\beta x)$
\end{formulabox}
\begin{procbox}{Variation of Parameters}
For $y''+p\,y'+q\,y = f(x)$ with solutions $y_1,y_2$:

$y_p = -y_1\!\int\!\frac{y_2 f}{W}dx + y_2\!\int\!\frac{y_1 f}{W}dx$

$W = y_1 y_2' - y_1' y_2$ (Wronskian)
\end{procbox}
\sheetsection{Sequences \& Series}
\begin{thmbox}{Ratio Test}
$L = \lim\left|\frac{a_{n+1}}{a_n}\right|$: \
$L<1$ converges, $L>1$ diverges, $L=1$ inconclusive
\end{thmbox}
\begin{thmbox}{Root Test}
$L = \lim\sqrt[n]{|a_n|}$: same criteria as ratio test
\end{thmbox}
\begin{thmbox}{Comparison Test}
$0 \leq a_n \leq b_n$: $\sum b_n$ conv $\Rightarrow$ $\sum a_n$ conv;
$\sum a_n$ div $\Rightarrow$ $\sum b_n$ div
\end{thmbox}
\begin{thmbox}{Alternating Series}
$\sum(-1)^n a_n$ converges if $a_n>0$, $a_n$ decreasing, $\lim a_n = 0$
\end{thmbox}
\sheetsection{Probability \& Statistics}
\begin{formulabox}{Expected Value \& Variance}
$\mathbb{E}[X] = \sum x_i\,p(x_i)$ or $\int x\,f(x)\,dx$

$\text{Var}(X) = \mathbb{E}[X^2] - (\mathbb{E}[X])^2$

$\text{Var}(aX+b) = a^2\text{Var}(X)$
\end{formulabox}
\begin{formulabox}{Common Distributions}
{\fontsize{5.5pt}{6.5pt}\selectfont
\textbf{Normal:} $f(x) = \frac{1}{\sigma\sqrt{2\pi}}e^{-(x-\mu)^2/(2\sigma^2)}$

\textbf{Exponential:} $f(x) = \lambda e^{-\lambda x},\ x\geq0$

\textbf{Poisson:} $P(X=k) = \frac{\lambda^k e^{-\lambda}}{k!}$

\textbf{Binomial:} $P(X=k) = \binom{n}{k}p^k(1-p)^{n-k}$
}
\end{formulabox}
\begin{thmbox}{Central Limit Theorem}
$X_1,\ldots,X_n$ i.i.d.\ with $\mu,\sigma^2$:
$\frac{\bar{X}-\mu}{\sigma/\sqrt{n}} \xrightarrow{d} N(0,1)$
\end{thmbox}
\sheetsection{Trigonometric Identities}
\begin{formulabox}{Pythagorean}
$\sin^2\!\theta + \cos^2\!\theta = 1$ \hfill
$1 + \tan^2\!\theta = \sec^2\!\theta$ \hfill
$1 + \cot^2\!\theta = \csc^2\!\theta$
\end{formulabox}
\begin{formulabox}{Sum / Difference}
\vspace{-1mm}
{\fontsize{5.5pt}{6.5pt}\selectfont
\begin{align*}
\sin(a\pm b) &= \sin a\cos b \pm \cos a\sin b \\
\cos(a\pm b) &= \cos a\cos b \mp \sin a\sin b \\
\tan(a\pm b) &= \frac{\tan a \pm \tan b}{1 \mp \tan a\tan b}
\end{align*}
}
\vspace{-2mm}
\end{formulabox}
\begin{formulabox}{Double Angle}
$\sin 2\theta = 2\sin\theta\cos\theta$

$\cos 2\theta = \cos^2\!\theta - \sin^2\!\theta = 2\cos^2\!\theta - 1 = 1 - 2\sin^2\!\theta$
\end{formulabox}
\sheetsection{Useful Inequalities \& Results}
\begin{formulabox}{AM-GM Inequality}
$\tfrac{a_1+\cdots+a_n}{n} \geq (a_1\cdots a_n)^{1/n}$ (equality iff all $a_i$ equal)
\end{formulabox}
\begin{thmbox}{Cauchy-Schwarz}
$|\langle\vect{u},\vect{v}\rangle| \leq \norm{\vect{u}}\norm{\vect{v}}$
\end{thmbox}
\begin{thmbox}{Triangle Inequality}
$\norm{\vect{u}+\vect{v}} \leq \norm{\vect{u}} + \norm{\vect{v}}$
\end{thmbox}
\sheetsection{Numerical Methods}
\begin{procbox}{Newton's Method}
$x_{n+1} = x_n - \frac{f(x_n)}{f'(x_n)}$ \quad (quadratic convergence)
\end{procbox}
\begin{formulabox}{Numerical Integration}
{\fontsize{5.5pt}{6.5pt}\selectfont
\textbf{Trap:} $\int_a^b f\approx \frac{h}{2}[f(x_0)+2\sum f(x_i)+f(x_n)]$

\textbf{Simpson:} $\int_a^b f\approx \frac{h}{3}[f_0+4\sum f_{\text{odd}}+2\sum f_{\text{even}}+f_n]$

where $h=(b-a)/n$
}
\end{formulabox}

% --- Footer ---
\vfill
\begin{center}
{\fontsize{4.5pt}{5.5pt}\selectfont\textcolor{exMed}{%
Exam Formula Sheet Template \textbar{} Maximum density for exam use \textbar{} Good luck!%
}}
\end{center}

\end{multicols}

\end{document}
