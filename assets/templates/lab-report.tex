\pdfoutput=1
\documentclass[11pt,a4paper]{article}

%=============================================================================
% ENCODING AND FONTS
%=============================================================================
\usepackage[utf8]{inputenc}
\usepackage[T1]{fontenc}

%=============================================================================
% PAGE LAYOUT AND TYPOGRAPHY
%=============================================================================
\usepackage[a4paper, margin=1in]{geometry}
% \usepackage{microtype}  % Uncomment if available for better typography

%=============================================================================
% HEADER AND FOOTER
%=============================================================================
\usepackage{fancyhdr}
\pagestyle{fancy}
\fancyhf{}
\lhead{\experimenttitle}
\rhead{\studentname}
\cfoot{\thepage}
\renewcommand{\headrulewidth}{0.4pt}

%=============================================================================
% MATH
%=============================================================================
\usepackage{amsmath}

%=============================================================================
% GRAPHICS AND FIGURES
%=============================================================================
\usepackage{graphicx}
% \usepackage{float}  % Uncomment if available for [H] placement specifier
% \usepackage[font=small,labelfont=bf]{caption}  % Uncomment if available for better captions
% \usepackage{subcaption}  % Uncomment if available for subfigures

%=============================================================================
% TABLES
%=============================================================================
% \usepackage{booktabs}  % Uncomment if available for professional table formatting
\usepackage{array}
% \usepackage{multirow}  % Uncomment if available for cells spanning multiple rows
% \usepackage{tabularx}  % Uncomment if available for auto-width tables

% Define booktabs commands if package is not available
\providecommand{\toprule}{\hline\hline}
\providecommand{\midrule}{\hline}
\providecommand{\bottomrule}{\hline\hline}

%=============================================================================
% LISTS
%=============================================================================
% \usepackage{enumitem}  % Uncomment if available for enhanced list formatting

%=============================================================================
% COLORS
%=============================================================================
% \usepackage{xcolor}  % Uncomment if available for colored links and text
% \definecolor{linkblue}{RGB}{0,51,153}

%=============================================================================
% UNITS, NUMBERS, AND UNCERTAINTIES
%=============================================================================
% If siunitx is available, uncomment the following lines for proper unit formatting
% \usepackage{siunitx}
% \sisetup{
%     separate-uncertainty=true,      % Format: 9.81 +- 0.02
%     multi-part-units=single,        % Units only once in uncertain values
%     detect-all
% }
% Alternative: Define simple macros for units without siunitx
\newcommand{\SI}[2]{#1\,#2}
\newcommand{\si}[1]{#1}
\newcommand{\num}[1]{#1}

%=============================================================================
% PLOTTING
%=============================================================================
% If pgfplots is available, uncomment for inline data plotting
% \usepackage{pgfplots}
% \pgfplotsset{compat=1.18}
% \usepgfplotslibrary{statistics}
% Alternative: Use \includegraphics to insert pre-generated plots

%=============================================================================
% HYPERLINKS (load near end)
%=============================================================================
\usepackage{hyperref}
\hypersetup{
    colorlinks=false,  % Set to true if xcolor is available
    % linkcolor=blue,  % Uncomment if xcolor is available
    % citecolor=blue,  % Uncomment if xcolor is available
    % urlcolor=blue,   % Uncomment if xcolor is available
    pdftitle={Lab Report},
    pdfauthor={Student Name},
    bookmarks=true
}

%=============================================================================
% DOCUMENT METADATA - CUSTOMIZE THESE
%=============================================================================
% Fill in your experiment information here
\newcommand{\experimenttitle}{Measurement of Gravitational Acceleration}
\newcommand{\coursename}{Physics 101: Mechanics Laboratory}
\newcommand{\studentname}{Jane Doe}
\newcommand{\studentid}{Student ID: 12345678}
\newcommand{\labpartner}{Lab Partner: John Smith}
\newcommand{\dateperformed}{Date Performed: January 15, 2025}
\newcommand{\datesubmitted}{Date Submitted: January 22, 2025}

%=============================================================================
% DOCUMENT
%=============================================================================
\begin{document}

% --- TITLE PAGE ---
\begin{titlepage}
    \centering
    \vspace*{2cm}

    {\LARGE\bfseries \experimenttitle\par}
    \vspace{1.5cm}

    {\Large \coursename\par}
    \vspace{2cm}

    {\large
    \studentname\\
    \studentid\\
    \vspace{0.5cm}
    \labpartner\\
    \vspace{1cm}
    \dateperformed\\
    \datesubmitted\\
    }

    \vfill

    % Optional: Add institution logo
    % \includegraphics[width=0.3\textwidth]{logo.png}

\end{titlepage}

% --- ABSTRACT ---
\begin{abstract}
\noindent
% Brief summary (150-250 words) of the entire experiment: objective, methods, key results, and conclusion
This experiment measured the acceleration due to gravity using a simple pendulum. Five different pendulum lengths ranging from $\SI{0.50}{m}$ to $\SI{1.50}{m}$ were tested, with period measurements taken for 10 oscillations each. The experimental value obtained was $g = \SI{9.81 +- 0.15}{m/s^2}$, which agrees with the accepted value of $\SI{9.81}{m/s^2}$ within experimental uncertainty. The primary sources of error were timing precision and small-angle approximation deviations.
\end{abstract}

\newpage

% --- TABLE OF CONTENTS (optional for longer reports) ---
% \tableofcontents
% \newpage

% --- INTRODUCTION / THEORY ---
\section{Introduction}
\label{sec:intro}

% Provide background on the physical principles being investigated
% State the objectives of the experiment
% Present relevant theoretical equations

The purpose of this experiment is to [state the objective clearly]. The physical principle being investigated is [describe the underlying theory].

\subsection{Theoretical Background}

% Present the theoretical framework and key equations
% Example for pendulum experiment:
% The period $T$ of a simple pendulum is given by:
\begin{equation}
    \label{eq:theory}
    T = 2\pi\sqrt{\frac{L}{g}}
\end{equation}
%
% where $L$ is the length of the pendulum and $g$ is the acceleration due to gravity.
% Rearranging for $g$:
\begin{equation}
    \label{eq:gravity}
    g = \frac{4\pi^2 L}{T^2}
\end{equation}

% Include any other relevant equations, assumptions, or theoretical considerations

% --- EXPERIMENTAL PROCEDURE ---
\section{Experimental Procedure}
\label{sec:procedure}

\subsection{Apparatus}

% List all equipment used with relevant specifications
The following equipment was used in this experiment:
\begin{itemize}
    \item [Equipment item 1 with specifications]
    \item [Equipment item 2 with specifications]
    \item [Equipment item 3 with specifications]
\end{itemize}

% Optional: Include a diagram of the experimental setup
% \begin{figure}[htbp]
%     \centering
%     \includegraphics[width=0.6\textwidth]{setup.png}
%     \caption{Experimental apparatus setup.}
%     \label{fig:setup}
% \end{figure}

\subsection{Methods}

% Describe the experimental procedure in clear, logical steps
% Write in past tense (what you did)
% Include enough detail for reproducibility

The experimental procedure was as follows:
\begin{enumerate}
    \item [First step of the procedure]
    \item [Second step of the procedure]
    \item [Third step of the procedure]
    \item [Continue with all steps...]
\end{enumerate}

% Note any special precautions, calibration procedures, or safety considerations

% --- DATA AND RESULTS ---
\section{Data and Results}
\label{sec:results}

\subsection{Raw Data}

% Present your measured data in well-formatted tables
% Use siunitx for proper alignment and uncertainty formatting

\begin{table}[htbp]
\centering
\caption{Experimental measurements with uncertainties.}
\label{tab:raw-data}
\begin{tabular}{@{}cccc@{}}
\toprule
\textbf{Length (m)} & \textbf{Time (s)} & \textbf{Period (s)} & \textbf{$g$ (m/s$^2$)} \\
\midrule
$0.50$ & $1.420$ & $1.420 \pm 0.010$ & $9.75 \pm 0.14$ \\
$0.75$ & $1.738$ & $1.738 \pm 0.010$ & $9.81 \pm 0.11$ \\
$1.00$ & $2.007$ & $2.007 \pm 0.010$ & $9.86 \pm 0.10$ \\
$1.25$ & $2.243$ & $2.243 \pm 0.010$ & $9.83 \pm 0.09$ \\
$1.50$ & $2.458$ & $2.458 \pm 0.010$ & $9.79 \pm 0.08$ \\
\bottomrule
\end{tabular}
\end{table}

% Include notes about how uncertainties were determined
\textbf{Note:} Uncertainties in period measurements represent the standard deviation of 5 trials. Uncertainties in $g$ were propagated using the formula in \ref{eq:gravity}.

\subsection{Data Analysis}

% Show calculations, data processing, and analysis
% Use proper mathematical notation

The average value of gravitational acceleration from the five measurements is:
\begin{equation}
    \label{eq:result}
    g_{\text{exp}} = \SI{9.81 +- 0.15}{\text{m/s}^2}
\end{equation}

% Example of uncertainty propagation
The uncertainty in $g$ was calculated using:
\begin{equation}
    \delta g = g \sqrt{\left(\frac{\delta L}{L}\right)^2 + \left(2\frac{\delta T}{T}\right)^2}
\end{equation}

\subsection{Graphical Analysis}

% Include relevant plots with error bars
% Show linear fits, trends, or relationships

% Option 1: If pgfplots is available, uncomment the tikzpicture below
% Option 2: Create plots externally (Excel, Python, etc.) and include with \includegraphics

\begin{figure}[htbp]
\centering
% Uncomment this tikzpicture if pgfplots package is installed:
% \begin{tikzpicture}
% \begin{axis}[
%     width=0.8\textwidth,
%     height=0.5\textwidth,
%     xlabel={Length $L$ (m)},
%     ylabel={Period Squared $T^2$ (s$^2$)},
%     grid=major,
%     legend pos=north west,
%     error bars/y dir=both,
%     error bars/y explicit,
% ]
% \addplot+[
%     only marks,
%     mark=*,
%     error bars/.cd,
%     y dir=both,
%     y explicit,
% ] coordinates {
%     (0.50, 2.016) +- (0, 0.028)
%     (0.75, 3.021) +- (0, 0.035)
%     (1.00, 4.028) +- (0, 0.040)
%     (1.25, 5.031) +- (0, 0.045)
%     (1.50, 6.042) +- (0, 0.049)
% };
% \addlegendentry{Experimental data}
% \addplot[thick, blue, domain=0.4:1.6,] {4.024*x};
% \addlegendentry{Linear fit: $T^2 = 4.024L$}
% \addplot[thick, red, dashed, domain=0.4:1.6,] {4.027*x};
% \addlegendentry{Theory: $T^2 = \frac{4\pi^2}{g}L$}
% \end{axis}
% \end{tikzpicture}

% Alternative: Use this placeholder or replace with your own graph image
\fbox{\parbox{0.8\textwidth}{
\centering
\vspace{3cm}
\textbf{[Insert graph: Period squared vs. Length]}\\[0.5em]
Use Excel, Python (matplotlib), or other tools to create your plot.\\
Save as PNG/PDF and include with: \texttt{\textbackslash includegraphics[width=0.8\textbackslash textwidth]\{graph.png\}}
\vspace{3cm}
}}

\caption{Period squared versus pendulum length with linear fit. The slope of the experimental line is $4.024 \pm 0.015$ s$^2$/m, giving $g = 9.81 \pm 0.15$ m/s$^2$.}
\label{fig:period-length}
\end{figure}

% --- ANALYSIS AND DISCUSSION ---
\section{Analysis and Discussion}
\label{sec:analysis}

\subsection{Comparison with Theoretical Value}

% Compare your results with accepted or theoretical values
% Calculate percent error

The accepted value for gravitational acceleration at sea level is $g_{\text{accepted}} = \SI{9.81}{\text{m/s}^2}$. The percent error is calculated as:
\begin{equation}
    \text{Percent Error} = \frac{|g_{\text{exp}} - g_{\text{accepted}}|}{g_{\text{accepted}}} \times 100\% = \frac{|9.81 - 9.81|}{9.81} \times 100\% = 0.0\%
\end{equation}

The experimental value agrees with the accepted value within the experimental uncertainty, indicating successful measurement.

\subsection{Sources of Error}

% Identify and discuss systematic and random errors
% Be specific about how each error source affects your results

The main sources of uncertainty in this experiment were:
\begin{itemize}
    \item \textbf{Timing precision:} Human reaction time in starting and stopping the timer contributed approximately $\SI{0.1}{s}$ uncertainty per measurement.
    \item \textbf{Amplitude variations:} Small deviations from the small-angle approximation may have introduced systematic errors.
    \item \textbf{Air resistance:} Damping effects were minimal but present, potentially affecting period measurements for longer trials.
    \item \textbf{Length measurement:} Uncertainty in measuring the pendulum length to the center of mass was approximately $\SI{0.5}{cm}$.
\end{itemize}

\subsection{Improvements}

% Suggest specific improvements for future experiments
Potential improvements for this experiment include:
\begin{itemize}
    \item Using photogate timing systems to eliminate human reaction time errors
    \item Conducting measurements in a vacuum chamber to eliminate air resistance
    \item Using a more precise method to determine the center of mass of the pendulum bob
\end{itemize}

% --- CONCLUSION ---
\section{Conclusion}
\label{sec:conclusion}

% Summarize the key findings (2-3 sentences)
% State whether objectives were met
% Comment on the agreement between experimental and theoretical values

This experiment successfully measured the acceleration due to gravity using a simple pendulum. The experimental value of $g = \SI{9.81 +- 0.15}{\text{m/s}^2}$ agrees well with the accepted value of $\SI{9.81}{\text{m/s}^2}$, with a percent error of approximately 1.5\%. The results validate the theoretical relationship in Equation~\ref{eq:theory} and demonstrate that the period of a simple pendulum depends on its length as predicted by theory.

% --- REFERENCES (optional) ---
\section{References}
\label{sec:references}

% List any references used (textbooks, manuals, papers)
% Use a consistent citation format

% Option 1: Manual bibliography
\begin{enumerate}
    \item Physics Laboratory Manual, Department of Physics, University Name (2025).

    \item Taylor, J.R.\ \textit{An Introduction to Error Analysis}, 2nd ed.\ University Science Books (1997).

    \item Halliday, D., Resnick, R., and Walker, J.\ \textit{Fundamentals of Physics}, 11th ed.\ Wiley (2018).
\end{enumerate}

% Option 2: BibTeX (uncomment and create .bib file)
% \bibliographystyle{plain}
% \bibliography{references}

% --- APPENDIX ---
\newpage
\appendix
\section{Raw Data Sheets}
\label{app:raw-data}

% Include raw data tables, calculations, or additional information

\begin{table}[htbp]
\centering
\caption{Complete trial-by-trial measurements for Length = $\SI{1.00}{m}$.}
\label{tab:detailed-data}
\begin{tabular}{@{}ccc@{}}
\toprule
\textbf{Trial} & \textbf{Time for 10 Oscillations (s)} & \textbf{Period (s)} \\
\midrule
1 & 20.05 & 2.005 \\
2 & 20.10 & 2.010 \\
3 & 20.08 & 2.008 \\
4 & 20.03 & 2.003 \\
5 & 20.09 & 2.009 \\
\midrule
Mean & 20.07 & 2.007 \\
Std. Dev. & 0.03 & 0.003 \\
\bottomrule
\end{tabular}
\end{table}

\section{Sample Calculations}
\label{app:calculations}

% Show detailed calculations for one representative example

\textbf{Example calculation for} $L = \SI{1.00}{m}$\textbf{:}

Given:
\begin{align*}
    L &= \SI{1.00 +- 0.005}{m} \\
    T &= \SI{2.007 +- 0.010}{s}
\end{align*}

Calculate $g$ using \ref{eq:gravity}:
\begin{align*}
    g &= \frac{4\pi^2 L}{T^2} \\
      &= \frac{4\pi^2 \times 1.00}{(2.007)^2} \\
      &= \frac{39.478}{4.028} \\
      &= \SI{9.81}{\text{m/s}^2}
\end{align*}

Uncertainty propagation:
\begin{align*}
    \delta g &= g \sqrt{\left(\frac{\delta L}{L}\right)^2 + \left(2\frac{\delta T}{T}\right)^2} \\
             &= 9.81 \sqrt{\left(\frac{0.005}{1.00}\right)^2 + \left(2\times\frac{0.010}{2.007}\right)^2} \\
             &= 9.81 \sqrt{(0.005)^2 + (0.00997)^2} \\
             &= 9.81 \times 0.0111 \\
             &= \SI{0.11}{\text{m/s}^2}
\end{align*}

Therefore: $g = \SI{9.81 +- 0.11}{\text{m/s}^2}$

\end{document}
